%############################################################################################
% Este modelo é iniciativa dos docentes do curso de Ciência da Computação da Unit/AL.
% Versão do modelo de monografia: 1.0.
% Criada por: Izaac Duarte de Alencar.
% Última modificação: 27/05/2021.
% Este modelo segue o padrão ABNT para formatação de texto científico.
%############################################################################################

\documentclass[
				12pt,                   % Tamanho da fonte.
				a4paper,                % Tamanho do papel.
				oneside,                % Impressão no verso.
				sumario=tradicional,    % Estilo tradicional de sumário
				english,
				brazil                  % O último idioma é o principal do documento
				]{abntex2}
				
%----------------
% PACOTES
%----------------		
% usar abnt-emphasize=bf para negritar os títulos em vez do itálico (usar dentro das configurações do pacote abntex2cite
\usepackage[alf,
			abnt-full-initials=yes,
			abnt-repeated-author-omit=yes,
			abnt-etal-text=default
			%bibjustif			% Justifica as referências
			]{abntex2cite}   	% Prepara Latex para citações ABNT (autor/data)
\usepackage{pslatex}            % Usa fonte Times New Roman.
%\usepackage{helvet}			% Importa fonte helvet/arial
%\usepackage[T1]{fontenc}		% Selecao de codigos de fonte.
\usepackage[utf8]{inputenc}		% Codificacao do documento (conversão automática dos acentos)
\usepackage{indentfirst}		% Indenta o primeiro parágrafo de cada seção.
\usepackage{microtype} 			% Para melhorias de justificação
\usepackage{graphicx} 			% Pacote para inserir imagens
\usepackage{array}           	% Tabela tamanho
\usepackage{float}				% Posicionamento tabela
\usepackage{caption}       		% para maior controle sobre legendas
\usepackage{hyperref}           % tratamento de URL
\usepackage{enumitem}
\usepackage{longtable}          % ambiente de tabelas longas com quebra de página
\usepackage{multicol,multirow}
\usepackage{lmodern}
\usepackage{pdfpages}           % adicionar pdf externo
\usepackage[brazil]{babel}
\usepackage{xurl}               % quebra de url
\usepackage{comment}            % ambiente para comentários em blocos


%----------------
% Configurações gerais
%----------------
\renewcommand{\familydefault}{\sfdefault}   % Aplica Fonte Helvet
\setlength{\parindent}{1.25cm}              % Recuo do parágrafo.
\setlength{\absparindent}{1.25cm}           % Recuo do resumo/abstract
\setlength{\parskip}{6pt}                   % Espaçamento entre parágrafos.
\setlrmarginsandblock{3cm}{2cm}{*}          % Margens superior e direita.
\setulmarginsandblock{3cm}{2cm}{*}          % Margens esquerda e inferior
\setsecheadstyle{\bfseries \normalsize}     % Estilo de sections
%\chapterstyle{article}				        % Estilo dos Capítulos

% Configuração para evitar separação de sílaba
\tolerance=1
\emergencystretch=\maxdimen
\hyphenpenalty=10000
\hbadness=10000
\hyphenchar\font=-1
\sloppy
% final da configuração

% ---------------
% Informações Cabeçalho
% ---------------
\title{Título do monografia: sub-título}
\author{Nome(s) do(s) Aluno(s)}                 % mais de um autor colocar ao final de cada nome \\
\orientador{Profº. Nome do Orientador}
\local{Maceió - Alagoas}
\data{2021.1}
\instituicao{CENTRO UNIVERSITÁRIO TIRADENTES}
\preambulo{Monografia apresentada como pré-requisito para obtenção de Bacharelado em Ciência da Computação.}


% ---------------
% Configurações especias
% ---------------
\usepackage{minted}                              % ambiente para códigos
\setsubsecheadstyle{\bfseries \normalsize}       % configuração de formatação (tamanho e estilo) das subsections

% ---------------
% Informações do PDF
% ---------------
\makeatletter

\hypersetup{
pdftitle={\@title},
pdfauthor={\@author},
pdfsubject={\@title},
colorlinks=true,            % false: boxed links; true: colored links
linkcolor=black,            % color of internal links
citecolor=black,            % color of links to bibliography
filecolor=magenta,          % color of file links
urlcolor=black,
bookmarksdepth=4
}

\makeatother

% Definição de diretório de imagens
\graphicspath{{IMG/}}

% ---------------
% Reescrevendo a capa
% ---------------
\makeatletter
\renewcommand{\imprimircapa}{%

\begin{capa}%

% A figura adicionada é o logotipo da Unit e não pode ser removido da pasta IMG
% Este ambiente é obrigatório na capa.
\begin{figure}[H]
\centering
\includegraphics[width=0.5\textwidth] {unit}\\
\end{figure}
\center

\ABNTEXchapterfont\bfseries\large\imprimirinstituicao\\
\ABNTEXchapterfont\large\ CURSO DE CIÊNCIA DA COMPUTAÇÃO\\


\vspace{3cm}

{\ABNTEXchapterfont\large\imprimirautor}

\vspace{3cm}

\begin{center}
\SingleSpacing
{\ABNTEXchapterfont\bfseries\Large\imprimirtitulo}
\end{center}

\vfill

{\large\imprimirlocal}

{\large\imprimirdata}
\vspace*{1cm}

\end{capa}

}
\makeatother

% ---------------
% Reescrevendo a folha de rosto
% ---------------
\makeatletter
\renewcommand{\folhaderostocontent}{

\begin{center}

{\ABNTEXchapterfont\large\imprimirautor}

\vspace{3.5cm}
\begin{center}
\SingleSpacing
\ABNTEXchapterfont\bfseries\Large\imprimirtitulo
\end{center}
\vspace{1.0cm}

\abntex@ifnotempty{\imprimirpreambulo}{%
\hspace{.45\textwidth}
\begin{minipage}{.5\textwidth}
\SingleSpacing
\fontsize{10pt}{\baselineskip}\selectfont \imprimirpreambulo

Orientador: \imprimirorientador
\end{minipage}%
\vspace*{\fill}
}%

\vspace*{\fill}

{\large\imprimirlocal}
\par
{\large\imprimirdata}
%\vspace*{1cm}

\end{center}

}
\makeatother

% ---------------
% Início do documento
% ---------------
\begin{document}

% ----------------------------------------------------------
% ELEMENTOS PRÉ-TEXTUAIS
% ----------------------------------------------------------

% ---------------
% Capa
% ---------------
\imprimircapa
% ---------------

% ---------------
% Folha de Rosto
% ---------------
\imprimirfolhaderosto
% ---------------

% ---------------
% Ficha catalográfica
% Deverá ser adicionada após a defesa, sendo solicitada sua confecção junto à biblioteca que enviará documento externo com o PDF
% Coloque a ficha catalográfica recebida, em PDF, na pasta PDF do modelo e mantenha com o nome fichacatalografica.pdf
% retire o comentário da linha % Comando para inserir arquivo PDF externo com a ficha catalográfica recebida pela biblioteca
% Certifique-se que do nome do arquivo PDF dentro da pasta PDF
\includepdf[pages=-]{PDF/fichacatalografica.pdf} e recompile.
% ---------------
%% Comando para inserir arquivo PDF externo com a ficha catalográfica recebida pela biblioteca
% Certifique-se que do nome do arquivo PDF dentro da pasta PDF
\includepdf[pages=-]{PDF/fichacatalografica.pdf}
% ---------------

% ---------------
% Folha de Aprovação
% ---------------
\begin{folhadeaprovacao}

\begin{center}

{\ABNTEXchapterfont\large\imprimirautor}

\vspace{3.5cm}


{\ABNTEXchapterfont\bfseries\Large\imprimirtitulo}


\end{center}

\vspace{1.0cm}

\hspace{.39\textwidth} 
\begin{minipage}{.5\textwidth}
\fontsize{10pt}{\baselineskip}\selectfont \imprimirpreambulo

Orientadora: \imprimirorientador
\end{minipage}


\vspace{2cm}


\begin{center}

Aprovado em \_\_\_\_\_\_ /\_\_\_\_\_\_/\_\_\_\_\_\_\_\_\_\_.

\vspace{1cm}

\assinatura{\imprimirorientador \\Orientador(a)} 

\assinatura{Examinador 1}

\assinatura{Examinador 2}

\end{center}

\end{folhadeaprovacao}
% ---------------

% ---------------
% Dedicatória
% ---------------
% Ambiente para escrita da dedicatória.
% Elemento opcional.
\chapter*{Dedicatória}

\begin{dedicatoria}

Dedico este trabalho...

\end{dedicatoria}
% ---------------

% ---------------
% Agradecimentos
% ---------------
% Ambiente para escrita de texto de agradecimentos
% Elemento opcional.
\begin{agradecimentos}

\end{agradecimentos}
% ---------------

% ---------------
% Epígrafe
% ---------------
% Ambiente para escrita de texto da epígrafe
% Elemento opcional.
\chapter*{Epígrafe}

\begin{epigrafe}

\vspace*{\fill}
% O texto deverá ser escrito dentro do ambiente flushright
\begin{flushright}

    \textit{O texto deverá ser escrito aqui!}

\end{flushright}

\end{epigrafe}
% ---------------

% ---------------
% Resumo
% ---------------
% Ambiente para escrita de resumo (português)

\begin{resumo}

O resumo deve ser escrito aqui!

\vspace{\onelineskip}
\textbf{Palavras-chaves}:
\end{resumo}


% Ambiente para escrita de resumo em inglês
\begin{resumo}[Abstract]

    \begin{otherlanguage*}{english}
    
    \vspace{\onelineskip}
    \noindent
    \textbf{keywords}:
    \end{otherlanguage*}

\end{resumo}
% ---------------

% ---------------
% Lista de Figuras
% ---------------
\listoffigures
% ---------------

% ---------------
% Lista de Tabelas
% ---------------
\newpage
\listoftables
% ---------------

% Inserir o SUMÁRIO
\newpage
\pdfbookmark[0]{\contentsname}{toc}
\tableofcontents*
\cleardoublepage

% ----------------------------------------------------------
% ELEMENTOS TEXTUAIS
% ----------------------------------------------------------
\textual

% ---------------
% Primeira Página
% ---------------
\pagestyle{simple}
%\input{primeirapagina}
% ---------------

%############################################################
% O comando \input adicionará o arquivo externo para montar o
% texto final do trabalho. Crie os arquivos dentro da pasta
% "CAPITULOS" e adicione-os em ordem como abaixo.
%############################################################

% ---------------
% INTRODUÇÃO
% ---------------
\pagestyle{simple}
% O comando \chapter abre um capítulo dentro do texto
\chapter{Introdução}


% Ambiente para adicionar Figura no texto
% O argumento "H" força o posicionamento da figura no local onde está colocado no texto
% O comando \caption adiciona um título da figura
% O comando \label adiciona uma chave de referência para citação da figura no texto
% O comando \includegraphics indica qual é a figura que será adicionada
% O comando \small permite colocar a texto com fonte menor para referência da figura
\begin{figure}[H]
\centering
\caption{Problema de Pesquisa.}
\label{fig:problemaPesquisa}
\includegraphics[width=\textwidth] {unit}\\
\small{Fonte: o autor.}
\end{figure}

A Figura~\ref{fig:problemaPesquisa} fala sobre

% O comando \section abre uma seção de texto dentro do capítulo
\section{Exemplo de seção}

\subsection{Exemplo de subsection}

% -------------------------
% ambiente de código
%\begin{minted}{java}
%\end{minted}
% -------------------------

% Criando citação no texto
% \cite{barbieri2019} - (BARBIERI, 2019)
% \citeauthor{barbieri2019} - BARBIERI (2019)

% Exemplo de listas no latex com ambiente itemize de listagem interna.
\begin{itemize}
    \item item 1
    \item item 2
    
    \begin{itemize}
        \item teste
    \end{itemize}
    
\end{itemize}

% Exemplo de lista numerada com o ambiente Enumerate
\begin{enumerate}
    \item item 1
    \item item 2
\end{enumerate}


% Criando citação no texto
% \cite{barbieri2019} - (BARBIERI, 2019)
% \citeauthor{barbieri2019} - BARBIERI (2019)
Essa citação é \textit{indireta} é faz parte do exemplo do cite da ABNT \cite{barbieri2019}.

Segundo \citeauthor{barbieri2019} (2019), isso é uma citação também.

% para citação direta
% \begin{citacao}
% é uma citação direta \cite[p. 32]{barbieri2019}. - (BARBIERI, 2019, p. 32)
% \end{citacao}
% Para citação direta, se faz necessário a indicação do número da página. Isto é realizado com a inclusão do [] antes da indicação da chave de citação. EX: \cite[p. 21]{chave/-citacao}
\begin{citacao}
é um exemplo de citação direta que preciso colocar de forma recuada no texto \cite[p. 32]{barbieri2019}. 
\end{citacao}
% ---------------

% ---------------
% CAPITULO1
% ---------------
\pagestyle{simple}
\chapter{Capítulo 1}

citando \cite{IDC2021}

citando \cite{Witkowski2021}

\url{https://www.marketwatch.com/story/videogames-are-a-bigger-industry-than-sports-and-movies-combined-thanks-to-the-pandemic-11608654990}

Exemplo de tabela: ambiente table

\begin{comment} % Este ambiente permite comentários em blocos.
Veja que não aparece no texto compilado. Utilize para fazer comentários longos.
\end{comment}

%=============================================================
% O ambiente table deve ser utilizado para tabelas pequenas, que não precisem ter quebras entre páginas.
\begin{table}[H]
    \centering
    \begin{small}
         \caption{Título da tabela}
         \begin{tabular}{c c}
         \hline
            Nome   &  Valor\\
         \hline
            Teste 02   &  7.85\\
            Teste 03   &  9.00\\
        \hline
         \end{tabular}
    \end{small}
    \label{tab:tituloTabela}
\end{table}
% ---------------

% ----------------------------------------------------------
% ELEMENTOS PÓS-TEXTUAIS (Referências, Glossário, Apêndices)
% ----------------------------------------------------------
\bibliography{library}


% ----------------------------------------------------------
% Apêndices
% ----------------------------------------------------------
% Elemento de produção própria do autor. Não é um elemento obrigatório

% ---
% Inicia os apêndices
% ---
\begin{apendicesenv}

% Imprime uma página indicando o início dos apêndices
\partapendices

%\include{APENDICES/revisaoSistematica}

\end{apendicesenv}


%----------------
% Fim do Documento
%----------------
\end{document}